\chapter{Summary, Conclusion, and Recommendations}

\addcontentsline{toc}{section}{Summary and Conclusion}
\section*{Summary and Conclusion}

The research was conducted to provide a rainfall amount forecast for Baguio City, using time series data from January 2001 to December 2011 gathered from PAG-ASA. 
Using Holt Winters simple exponential smoothing technique through the statistical software R-statistics, the data gathered were used to create a model to produce a forecast for the year 2012. 

Through the auto-correlation function, it was found out that the model derived through Holt-Winters exponential smoothing might be improved upon. However, the Ljung-Box test showed that there is little evidence of non-zero autocorrelations in the in-sample forecast
errors, and the distribution of forecast errors seems to be normally distributed with mean zero. These suggest
that the exponential smoothing, Holt-Winters method provides an adequate predictive model for  rainfall, which
probably cannot be improved upon. Furthermore, the assumptions that the 80\% and 95\% predictions intervals
were based upon (that there are no autocorrelations in the forecast errors, and the forecast errors are normally
distributed with mean zero and constant variance) are probably valid.

%It was also shown that there was no significant difference between the actual and forecasted values. 

\addcontentsline{toc}{section}{Recommendations}
\section*{Recommendations}
The proponents recommend that a better statistical study be done to further prove the validity of the forecasts. It is also possible that a better and more accurate forecast be produced through other statistical models. It is also recommended that this research be done on other weather factors such as temperature, air pressure, humidity, and number of rainy days.