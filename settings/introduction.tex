\chapter{Introduction}

%\begin{figure}[ht]
%  \centering %%used to center the figure
  %% The width option is used to rescale the figure into the proper
  %% fraction of the \columnwidth
%  \includegraphics[width=0.53\columnwidth]{figures/small.jpg}\\
%  \caption[This appears on List of Figures]
%{This is the caption that appears at the bottom of the figure}
%  \label{small}
%\end{figure}

%\pagebreak % This command forces succeeding text to appear on the next page

%\ref{silofailure}.

%\newpage % same behavior as \pagebreak

	Precipitation is any form of humidity that falls from the clouds in the air to the exterior of the Earth. Since precipitation refers to the liquid quantity and how much of it is isolated on the Earth within a given time, it is measured in volumes and concentration of precipitation on specific areas where the study is focused on. (Ramsey, 1998)
	
	Rain is a group of droplets that tends to fall towards the land of the Earth. The cloud cannot already include the amount of cloud droplets present within it that is why the cloud needs to release these droplets and when they are released, these droplets are already called as rain. Rain is the only type of liquid precipitation, as opposed to non-liquid types of precipitation, which are sleet, snow, and hail. A presence of a thick layer of our atmosphere is needed by rain to maintain temperatures above the melting point of water on the surface of the Earth. When ice crystals within a specific cloud collide against each other, precipitation is formed. Ice crystals have different shapes. There are oblate crystals, round-shaped crystals, and crystals that look like a small sphere. 
(Ramsey, 1998) The major cause of rain production is moisture contrasts that are commonly called as weather fronts and some moisture moving along the zones of temperature. Based on the location of the Philippines, this country only experiences rain, drizzle, and hail among the other types of precipitation. (“Earth Science: The Philippines in Focus,” 1983)

	Since the precipitation is measured in volumes of water in a specific area, the best way to measure the amount of precipitation is to gather all fallen liquid on a specific area with the use of waterproof walls and bases to see how high the water would increase from ground level.  An instrument used in this process with a similar mechanism is the rain gauge.  The rain gauge is the most widely used weather instrument in measuring precipitation.  The rain gauge is composed of a funnel and a cylindrical container where the water accumulates and is collected. However, a rain gauge is most effective when used in a perfectly flat area with its surroundings of the same level.  When used in mountainous regions or areas with uneven ground levels, either the measurements would be inaccurate or multiple rain gauges must be used for each ground level. Rainfall varies in amounts depending on the altitude.  The measurements on a rain gauge are only applicable on a fairly small radius or area around it, any data that would need more information about the amount of rainfall on a specific radius would be erroneous.
	
	The most common rain detector used in electronic weather stations is the “tipping bucket” type of rain sensor.  This fascinating type of technology uses two small “buckets” mounted on a swivel. The tiny buckets are manufactured with tight tolerances to guarantee that they hold an exact quantity of precipitation.  The tipping bucket assembly is to be found underneath the rain collector, which funnels the precipitation to the buckets.  As rainfall fills the tiny bucket, it becomes overbalanced and tips down, emptying itself as the other bucket pivots into place for the next reading. The action of each tipping episode triggers a small control that activates the electronic circuitry to transmit the count to the indoor console.  On a wireless rain gauge, records are transmitted through a radio signal. (“WW2010,” 2003)
	
	These methods aforementioned are some methods that PAGASA Weather Station is implementing to gather records of rainfall during the entire day, where they collect data every after three hours starting at two in the morning until eleven in the evening.
	
	The PAGASA Weather Station, also recognized as Philippine Atmospheric, Geophysical and Astronomical Services Administration, is a nationwide institution of the Philippines that provides warnings about flood and typhoon.  They also provide a lot more services like public advisories and forecasts concerning the up to date weather report of the country.  PAGASA furthermore provides meteorological, astronomical, and climatological information for the security of life and property of the Filipino people.  This government agency started operating on the 8th of December in 1972.
	
	This agency has a mandate that states that they need to provide protection against natural calamities to ensure the safety of the Filipino citizens, well-being and economic security of all the people, and for promotion of national progress.
	
	Residents in the Philippines would expect to have a huge amount of rainfall every month of the year.  The rainy season starts on the end of May and ends on late November or early December. (“Earth Science: The Philippines in Focus,” 1983)
	
	In Batanes, Northeastern Luzon, Western part of Camarines Norte, Camarines Sur, Albay, Bondoc Peninsula, Eastern Mindoro, Marinduque, Western Leyte, Northeastern Cebu, Bohol, and most of the Central and Southern Mindanao experience rainfall that us more or leass evenly distributed all throughout the year. (“Earth Science: The Philippines in Focus,” 1983)
	
	Upon observing the rainfall pattern in Baguio City, the proponents also observed some factors that could massively affect the rainfall in our city.  One factor that would affect the rainfall pattern of Baguio City is the season.  According to some references, high precipitation occurs during the humid season of the year while low precipitation occurs during the dry season of the year. Since the city is located at a high altitude, the elevation could also affect the pattern of rainfall that will occur.  Mountains affect the amount of rainfall.  Rains fall more often on the slopes facing the wind than on the slope away from the wind. The reason is that a wind hitting the side of the mountain tends to rise along the slope reaching heights of low temperature.  There, the moisture in the wind condenses to form rain.  By the time it reaches the other side of the mountain there is not enough amount of moisture to further condense.  The eastern coastal areas generally receive more rainfall than the western parts.  The eastern areas have high rainfall from October to March when the monsoon blows over the country.  For the Philippines as a whole, June to December are the rainy months while January to May are the dry months. (“Earth Science: The Philippines in Focus,” 1983)

\addcontentsline{toc}{section}{Background of the Study}
\section*{Background of the Study}

Baguio City is a highly urbanized city located in the province of Benguet.  It has an altitude of 1610 meters and covers a total land are of 57.5 \si{\square\kilo\meter}. Landslide and flashflood occurrences are highly unpredictable in some areas of the city because rainfall amount does not have a recognizable pattern.  Thus, many parts of Baguio City are suffering from landslides and flashfloods during periods of unpredicted heavy rainfall.  In order to increase the safety, awareness against such environmental disasters and an analysis regarding the rainfall patterns of Baguio City has to be done to provide basic information.

	Such study has been done in different countries such as Northeastern Thailand, India, and Australia.  The necessary data shall be collected from the weather station of Baguio City to produce a forecast for the amounts of rainfall every year.
	
\addcontentsline{toc}{section}{Statement of the Problem}
\section*{Statement of the Problem}

	The amount of precipitation in Baguio City has become very unpredictable, to a point where landslides and flashfloods have become unforeseeable. The aim of the study is to provide a basic forecast about how much precipitation would fall on Baguio City on the succeeding year, based on the ten-year data gathered from PAGASA.

%
%\addcontentsline{toc}{section}{Scope and Limitation of the Study}
%\section*{Scope and limitation of the study}
%
%\addcontentsline{toc}{section}{Time and place of study}
%\section*{Time and place of study}
\addcontentsline{toc}{section}{Significance of the Study}
\section*{Significance of the study}

This study can be a reference for preparations for certain agricultural activities like cultivating, planting, and harvesting.  It can also be a reference as a precautionary measure for flash floods and landslides.

\addcontentsline{toc}{section}{Scope and Delimitation}
\section*{Scope and Delimitation}

The study is limited only to analyzing rainfall amounts and no other weather factors.  The study has been limited to only analyzing the rainfall amounts in Baguio City because it ensures the safety of the researchers and it gives the easiest access to the needed data. This study was also limited to determining the coefficients of the Holt-Winters additive seasonal model and not all the set of the equations in the model. This study is further delimited to the prediction of the average monthly rainfall for the months of 2012.

\addcontentsline{toc}{section}{Definition of Terms}
\section*{Definition of Terms}

For clearer understanding of terms used in this study, below are the operational definitions of the terms used in this research paper.

\textbf{Time series analysis} concerns the analysis of data collected over time. Usually
the intent is to discern whether there is some pattern in the values collected
to date, with the intention of \textit{short term} forecasting

\textbf{Seasonality} is defined to be the tendency
of time-series data to exhibit behavior that repeats itself over regular periods.

\textbf{Additive seasonality} shows steady seasonal fluctuations, regardless of the overall level of the series.

\textbf{Multiplicative seasonality}, the size of the seasonal fluctuations vary,
depending on the overall level of the series.

\textbf{Exponential smoothing} is a procedure for continually revising a forecast in the light of more recent experience. Exponential Smoothing assigns exponentially decreasing weights as the observation get older. In other words, recent observations are given relatively more weight in forecasting than the
older observations.

\textbf{Forecasting} is the process of making projections about future performance on the basis of historical and current data.

\textbf{Holt-Winters} is a set of equations which handle time series data that show trend, seasonality, and a random effects.

\textbf{Additive Seasonal Model} is the Holt-Winters model used when the data exhibits Additive seasonality. In this model, we assume that the time series is represented by the model

\begin{equation}
y_t = a + b t + S_t + \epsilon_t
\end{equation}
where,
\begin{flushleft}
$y_t$ response of interest at time $t$\\
$a$ is the base signal also called the permanent component\\
$b$ is a linear trend component\\
$S_t$ is a additive seasonal factor\\
$\epsilon_t$ is the random error component\\
Let the length of the season be $L$ periods.\\
\end{flushleft}
The seasonal factors are defined so that they sum to the length of the season, that is

\begin{equation}
\sum_{1\le t\le L} S_t = 0.
\end{equation}
The trend component $b$ if deemed unnecessary, maybe deleted from the model.

The application of the model and further description of the rest of the set of equations in the model can be found in Kalekar (2004, p 7).

\textbf{Precipitation} is a deposit on earth of hail, mist, rain, sleet, or snow. It is also the quantity of water deposited.

\textbf{Rainfall} is the amount of precipitation usually measured by the depth in millimeters.

\textbf{Rainy day} refers to the period where precipitation occurs at any time of the day.